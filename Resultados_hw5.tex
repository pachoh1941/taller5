\documentclass{article}
\usepackage[utf8]{inputenc}
\usepackage{graphicx}
\title{Resultados Tarea 5}
\author{Juan Francisco Hurtado P\'erez}
\date{25 de mayo de 2017}

\begin{document}

\maketitle

\section{Canales i\'onicos}
Se encontr\'o el c\'irculo m\'aximo que cabe al interior de uno de los poros de canales i\'onicos, simulado en dos dimensiones por puntos, y cuyas coordenadas est\'an en el archivo de entrada.\\
Esto se logr\'o empleando un m\'etodo similar al Metropolis-Hastings, se utiliz\'o un par\'ametro $\alpha$ que permite discernir entre los avances de la caminata, y obtener el radio m\'aximo entre alguno de los pares de puntos del archivo.\\
A continuaci\'on, se muestran os par\'ametros encontrados en el procedimiento y la gr\'afica del c\'iculo que se obtiene de este modelo bi-dimensional.\\
\begin{figure}
    \centering
    \includegraphics[scale = 0.5]{circulo.png}
    \caption{Ilustracion 2D del canal i\'onico. Par\'ametros obtenidos y c\'irculo m\'aximo.}
\end{figure}
\section{Carga de un circuito RC}
Se utiliz\'o el m\'etodo de aproximaci\'on Bayesiana basada en MCMC y el m\'etodo Metropolis-Hastings para obtener los par\'ametros que mejor se ajustaran a los datos de carga de un capacitor, dados en un archivo de texto.\\
La ecuaci\'on que carcateriza esta carga viene dada por el modelo:\\
\begin{equation}
    q(t) = Q_{max}(1-e^{-t/RC})
\end{equation}
Para ello, se visualizaron los datos grafic\'andolos para tener una idea de los par\'ametros, y hacer la aproximaci\'on inicial (el primer lanzamiento de la caminata aleatoria). De all\'i en adelante, el c\'odigo generado se encarg\'o de estimar lo par\'ametros. \\
En seguida se muestran los resultados de las caminatas para cada uno y la gr\'afica del modelo contrastada con con los datos brutos del archivo de texto.\\
\begin{table}
    \centering
    \begin{tabular}{|cc|}
        \hline
        \includegraphics[scale=0.4]{graf_RvLike} & \includegraphics[scale=0.4]{histR} \\
        \hline
        \includegraphics[scale=0.4]{graf_CvLike} & \includegraphics[scale=0.4]{histC} \\
        \hline
    \end{tabular}
    \caption{Estad\'istica de obtenci\'on de los par\'ametros}
\end{table}\\
Finalmente, utilizando los par\'ametros encontrados y los datos de tiempo (primera colunma/variable independiente) del archivo de texto se obtuvo la curva que modela los datos.\\
\begin{figure}
    \centering
    \includegraphics[scale = 0.5]{graf_modelo.png}
    \caption{Resultado final}
\end{figure}

\end{document}
